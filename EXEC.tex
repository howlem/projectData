\documentclass{article}

\usepackage{apacite}

\title{Agriculture in Africa, 2016}
\author{Melissa Howlett}

\usepackage{Sweave}
\begin{document}
\Sconcordance{concordance:EXEC.tex:EXEC.Rnw:%
1 7 1 1 0 13 1 1 3 2 0 1 4 2 0 2 1 1 3 1 0 2 1 1 4 2 0 2 1 1 3 1 0 2 1 %
1 4 2 0 2 1 1 3 1 0 2 1 1 3 1 0 2 1 1 4 2 0 1 3 1 0 2 1 1 3 1 0 1 1 1 3 %
1 0 1 3 1 0 1 4 6 0 1 3 35 1}

\maketitle

\begin{abstract}
This paper uses Food and Agriculture Organization of the United Nations (FAO) data to examine the relationship between yield, production, and area harvested in Africa. 
\end{abstract}

\section{Introduction}\label{intro}
This is my intro to my great paper, I will explain the cool things I can do with my new `computational thinking' powers combined with some Latex.

This is my nice intro to my great paper, 
I will explain the cool things I can do with my new `computational thinking' powers combined with some Latex.

\begin{Schunk}
\begin{Sinput}
> #get data
> faostat=read.csv("https://github.com/howlem/projectData/raw/master/Production_Crops_E_All_Data.csv",stringsAsFactors = F)
> ##YIELD##
> #just keep yield data
> yield_1=faostat[faostat$'Element'!='Production',]
> yield=yield_1[yield_1$'Element'!='Area harvested',]
> head(yield)
> #aggregate by country
> yield=aggregate(yield$'Y2016',by=list(yield$'Area'),sum)
> names(yield)=c('country','TotYield2016')
> yield
> ##PRODUCTION##
> #just keep production data
> prod_1=faostat[faostat$'Element'!='Yield',]
> prod=prod_1[prod_1$'Element'!='Area harvested',]
> head(prod)
> #aggregate by country
> prod=aggregate(prod$'Y2016',by=list(prod$'Area'),sum)
> names(prod)=c('country','TotProd2016')
> prod
> ##AREA HARVESTED##
> #just keep area harvested data
> area_harv_1=faostat[faostat$'Element'!='Yield',]
> area_harv=area_harv_1[area_harv_1$'Element'!='Production',]
> head(area_harv)
> #aggregate by country
> area_harv=aggregate(area_harv$'Y2016',by=list(area_harv$'Area'),sum)
> names(area_harv)=c('country','TotAreaHarv2016')
> area_harv
> #merge yield, production, and area harvested data into one dataset
> combinedYP=merge(yield,prod)
> combinedYPAH=merge(combinedYP,area_harv)
> head(combinedYPAH)
> ##MAP DATA##
> #get zip file from github
> compressedMap="https://github.com/howlem/projectData/blob/master/worldMap/worldMap.shp?raw=true"
> #unzip the file
> temp=tempfile()
> download.file(compressedMap,temp)
> unzip(temp)
> #select the map (shp file) needed
> thefile=file.path('worldMap','worldMap.shp')
> worldMap <- rgdal::readOGR(thefile,stringsAsFactors=F) # use the names
> #only keep African countries
> worldMap=worldMap[worldMap$REGION==2,]
> #rename "NAME" column to "country"" so can merge
> names(worldMap)[names(worldMap)=='NAME'] <- 'country'
> #merge the two datasets into one, by country
> YPAHforMap <- merge(worldMap,combinedYPAH,
+                     by=c("country"))
> 
\end{Sinput}
\end{Schunk}

% <<location,echo=FALSE, fig=TRUE,eval=TRUE,height=2.6>>=
% library(RColorBrewer)
% library(classInt)
% library(ggplot2)
% 
% #making numeric
% as.numeric(data$'TotYield2016')
% 
% #identify variable to plot
% varToPlot=YPAHforMap$TotYield2016
% 
% #plot yield data on map of Africa
% numberOfClasses = 8
% colorForScale='YlGnBu'
% colors = brewer.pal(numberOfClasses, colorForScale)
% intervals <- classIntervals(varToPlot, numberOfClasses, 
%                             style = "quantile",
%                             dataPrecision=2)
% colorPallette <- findColours(intervals, colors)
% 
% legendText="Total Yield of all crops in 2016"
% shrinkLegend=0.5
% title="Total 2016 Yield by Country in Africa"
% 
% plot(worldMap,col='gray',main=title)
% plot(YPAHforMap, col=colorPallette, border='grey',add=T)
% 
% legend('topright', legend = names(attr(colorPallette, "table")), 
%        fill = attr(colorPallette, "palette"), cex = shrinkLegend, 
%        bty = "n",
%        title=legendText)
% @


\end{document}
